% Oliver Dixon's Curriculum Vitae (Public Copy)
% OWD, 2023

% TODO: Add module grades before submitting to FastTrack;
% TODO: My home address will probably have changed by the time I start applying;

\loop\iftrue
        \errmessage{Are you sure that you want to TeX my CV? :)}%
        \repeat
\pausing1

\documentclass[final]{article}

\usepackage[a4paper]{geometry} % Use correct margin sizes for A4 paper
\usepackage{setspace} % Custom line spacing
\usepackage[en-GB]{datetime2} % Regional formatting of \today
\usepackage{graphicx, xcolor} % Graphics support for the social icons
\usepackage{fancyhdr, lastpage} % Header and footer formatting
\usepackage{lmodern} % Correct font-sizing

\usepackage[
        angle = 90,
        anchor = lm,
        hpos = 4em,
        color = red!40,
        fontsize = .05\paperwidth%
]{draftwatermark}

\usepackage[
        colorlinks = true,
        allcolors  = blue
]{hyperref}

% We cannot use \textbf in a key-value specification, so we set the text here.
\DraftwatermarkOptions{text=\textbf{Draft CV}}

% \newcommand{\smallmargin}[1]{\marginpar{\raggedright\footnotesize #1}}
% \newcommand{\todomark}[1]{{\color{red}\textbf{[TODO:\@ #1]}}}

\setstretch{1.1} % Marginally greater line spacing improves readability
\addtolength{\textheight}{4em} % Increase the writable vertical area

% Do not indent new paragraphs. This is usually poor style, but since we are
% preferring lists (which have their own vertical spacing) over paragraphs for
% separation of prose, a non-zero \parindent would require excessive \noindent
% invocations.
\setlength{\parindent}{0pt}

\newlength{\socialsep}
\setlength{\socialsep}{.3\baselineskip}

% Included social icons may have a page group defined; these will not clash with
% the master document page groups, so it is safe to suppress this warning.
\pdfsuppresswarningpagegroup=1

% Remove number printing; preserve the hyperref index and TOC
\renewcommand{\thesection}{\hspace{-1em}}

% Make bullet points slightly smaller
\renewcommand{\labelitemi}{$\vcenter{\hbox{\tiny$\bullet$}}$}

% The 'body' style decorates all pages past the first page.
\renewcommand{\footrulewidth}{.4pt}
\makeatletter
\fancypagestyle{body}{%
	\fancyhf{}
        \renewcommand{\headrulewidth}{\footrulewidth}
        \fancyfoot[R]{%
                \hypersetup{linkcolor=black}%
                Page \thepage\ of~\pageref{LastPage}%
        }
}
\makeatother

% The 'title' style decorates only the title page; only the header is omitted.
\fancypagestyle{title}{%
        \fancyhead{}
        \renewcommand{\headrulewidth}{0pt}
}

\title{Curriculum Vitae}
\author{Oliver Dixon}

% \sociallink: include a social link with a small icon
%
%       #1: The platform identifier, used for resolving the icon file path;
%       #2: The display text for the hyperlink;
%       #3: The hyperlink destination.
%
\newcommand{\sociallink}[3]{%
        \raggedleft%
        {%
                \large
                \href{#3}{#2}%
                \hspace{.8em}%
                \begingroup
                \setbox0=\hbox{\includegraphics[width=.8em]{icons/#1}}%
                \parbox{\wd0}{\box0}
                \endgroup
        }%
        \par
}

% \yorkemail: expand to a clickable mailto URL of a York e-mail address
%
%       #1: The display name;
%       #2: The York username.
%
\newcommand{\yorkemail}[2]{%
        #1
        \textless%
        \href{mailto:#2@york.ac.uk}{\texttt{#2@york.ac.uk}}%
        \textgreater%
}

% \githublink: expand to a clickable link to the ... GitHub ...\ site
% \gistlink  : ... GitHub Gist ...
% \weblink   : ... York HTTP instance ....
%
%       #1: The URL suffix;
%       #2: The display name.
%
\newcommand{\githublink}[2]{\href{https://github.com/oliverdixon/#1}{#2}}
\newcommand{\gistlink}[2]{\href{https://gist.github.com/oliverdixon/#1}{#2}}
\newcommand{\weblink}[2]{\href{https://www-users.york.ac.uk/~od641/#1}{#2}}

% \midtilde: display a vertically centered text-mode tilde
\newcommand{\midtilde}{\raisebox{.5ex}{\texttildelow}}

% \textline: draw a black horizontal rule across the writeable page width
\newcommand{\textline}{\par\rule{\textwidth}{\footrulewidth}}

\begin{document}
%
\thispagestyle{title}
\pagestyle{body}
%
% MAJOR SECTION: TITLE AND METADATA HEADER
%
\phantomsection%
\makeatletter
\addcontentsline{toc}{chapter}{\@author: \@title}%
\makeatother
\begin{center}
        \Large
        \makeatletter
        \textbf{\@author} \hfill%
        \sociallink{github}{github.com/oliverdixon}%
                {https://github.com/oliverdixon}
        \makeatother
        \large

        \vspace{\socialsep} \hfill%
        \sociallink{internet}{www-users.york.ac.uk/\midtilde{}od641}%
                {https://www-users.york.ac.uk/~od641}

        \vspace{\socialsep}
        7 Monkton Road, York, YO31 9AJ \hfill%
        \sociallink{telephone}{+44 (0) 7341416944}{tel:+447341416944}

        \vspace{\socialsep}
        Mathematics \& Computer Science Student \hfill%
        \sociallink{email}{od641@york.ac.uk}{mailto:od641@york.ac.uk}
\end{center}
\textline%
%
% MAJOR SECTION: SUMMARY
%
\phantomsection%
\addcontentsline{toc}{section}{Summary}%
\vspace{2em}
I am a Mathematics and Computer Science joint honours undergraduate student at
The University of York, currently seeking a Year in Industry while averaging
over 90\% in academic disciplines.  In addition to having a keen and
long-established interest in Pure Mathematics and Theoretical Computer Science,
I am well-versed in a diverse range of practical technologies including C, Java,
Haskell, \LaTeX{}, \textit{Mathematica}, and a variety of scripting languages.
These act in complement to my demonstrable abilities in strong leadership,
communication, and organisation.

%
% MAJOR SECTION: EDUCATION
%
\section{Education}
\begin{itemize}
        \item \textbf{2022--Present, University of York (MMath)}
        \begin{itemize}
                \item MMath Mathematics and Computer Science, with a Year in
                Industry.

                \textit{In progress, with an expected graduation date of
                September 2027.}

                \item Mathematics: Algebra: ---\%; Calculus: ---\%; Mathematical
                Skills: ---\%.

                \item Computer Science: Software Part 1: 95\%; Software Part 2:
                ---\%; Theory: ---\%.
        \end{itemize}
        \item \textbf{2018--2020, Wakefield College (A-Levels)}
        \begin{itemize}
                \item A-level Mathematics (\textbf{A})
                \item A-level Further Mathematics (\textbf{A})
                \item A-level Computer Science (\textbf{A})
                \item A-level History (\textbf{A})
        \end{itemize}
        \item \textbf{2015--2018, Outwood Grange Academy (GCSEs)}
        \begin{itemize}
                \item 10 GCSEs achieved at grades \textbf{A}--\textbf{C},
                including English and Mathematics.
        \end{itemize}
\end{itemize}

%
% MAJOR SECTION: EXPERIENCE AND TECHNICAL SKILLS
%
\section{Experience and Technical Skills}
Whilst I am constantly concerned with the continual development of my commercial
and managerial aptitude, a wide variety of experiences have allowed me to
ascertain a significant set of skills bound to be invaluable to any employer:
\begin{itemize}
        \item Expertise---substantiated by over a decade's experience---in the C
        programming language. Some expositions of relevant work can be found on
        GitHub and my University website:
        \begin{itemize}
                \item \weblink{pss}{Programming Support Sessions}: During my
                tenure as the Technical Officer of The University of York
                Electronics Society (\textit{ShockSoc}) in the Autumn and Spring
                terms of 2022, I compiled and delivered a full course of the C
                programming language to cohorts of undergraduate students
                studying Electronic Engineering.

                Throughout the orchestration of this project, in the interests
                of securing a successful execution, I demonstrated impeccable
                long-term organisational and leadership skills: this included
                administration of the Societal and Departmental assets, and
                management of all interested parties.

                \item \githublink{calculator-demo}{An Arithmetic Expression
                Evaluator}: This project includes a reference implementation of
                the Shunting Yard algorithm, stack-based evaluation, and various
                other commonly techniques including memory management and the
                implementation of abstract data types.  Implementations in C and
                Java are included. This project is under active development,
                with the eventual goal of producing a more generalised computer
                algebra system.

                \item \weblink{fractal}{A Hardware-Accelerated Fractal
                Renderer}: This product, coupled with a large report, was
                written for an A-Level Computer Science qualification, however
                continues to be maintained in a limited capacity. Using C,
                OpenGL, and assembly language on a 64-bit Linux platform, the
                final product was capable of rendering parametrisable
                self-similar two-dimensional fractals to an X11 window using
                distributed computing paradigms.

                \item \gistlink{}{GitHub Gists Library}: A collection of modular
                and extensible implementations of canonical algorithms and
                abstract data structures which pervade classical Computer
                Science and Software Engineering. Selected contributions include
                a Hamming Weight benchmarking toolkit, an optimised
                implementation of a string-formatter, and a visual demonstration
                of dynamic allocation, intended for the education of
                undergraduate Computer Science and Electronic Engineering
                students.
        \end{itemize}
        \item A matured acquaintance with the \TeX\ and \LaTeX\ typesetting
        systems. In addition to the aforementioned Programming Support Session
        scripts, which were wholly set with \LaTeX, I have an array of similarly
        technical document-preparation projects:
        \begin{itemize}
                \item \githublink{MS1GP}{\textit{Mathematical Skills} Group
                Project}: A small set of documents covering Category Theory,
                with a particular focus on its parallels to functional
                programming and The $\lambda$-Calculus. The repository
                includes a report and large \textit{Beamer} presentation,
                throughout which abstract concepts are substantiated with
                concrete examples in the Haskell programming language. A
                collection of Bash shell scripts was also developed to
                facilitate effective management and deployment of this project.

                \item \githublink{cv}{Curriculum Vitae}: The sources of my CV
                demonstrate understanding of imperative and declarative design
                principles for non-trivial typesetting tasks.
        \end{itemize}
        \item A familiarity with the Wolfram \textit{Mathematica} technical
        computing system, particularly in the research areas of Automated
        Theorem-Proving and Distributed Workflows. A small cache of
        \textit{Mathematica} notebooks can be found on
        \weblink{misc\_mma}{my University web instance}.
\end{itemize}
%
% MAJOR SECTION: References
%
\section{References}
References are available upon request.
%
\end{document}

