% Oliver Dixon's Curriculum Vitae (Public Copy)
% OWD, 2023

% \loop\iftrue
%   \errmessage{Are you sure that you want to TeX my CV? :)}%
%   \repeat
% \pausing1

\documentclass[final]{article}

\usepackage[a4paper, margin=30mm]{geometry}
\usepackage[en-GB]{datetime2}
\usepackage{graphicx, xcolor, titlesec, tabularx}
\usepackage[colorlinks = true, allcolors = blue]{hyperref}

% Included social icons may have a page group defined; these will not clash with
% the master document page groups, so it is safe to suppress this warning.
\pdfsuppresswarningpagegroup=1

\renewcommand\baselinestretch{1.1}
\setlength\parindent{0pt}

\titleformat{\section}{\centering\Large\scshape}{}{0pt}{}
\renewcommand\labelitemi{$\vcenter{\hbox{\tiny$\bullet$}}$}

\title{Curriculum Vitae}
\author{Oliver Dixon}

% \yorkemail: expand to a clickable mailto URL of a York e-mail address
%
%   #1: The display name;
%   #2: The York username.
%
\newcommand{\yorkemail}[2]{%
  #1
  \textless%
  \href{mailto:#2@york.ac.uk}{\texttt{#2@york.ac.uk}}%
  \textgreater%
}

% \githublink: expand to a clickable link to the ... GitHub ...\ site
% \gistlink  : ... GitHub Gist ...
% \weblink   : ... York HTTP instance ....
%
%   #1: The URL suffix;
%   #2: The display name.
%
\newcommand\githublink[2]{\href{https://github.com/oliverdixon/#1}{#2}}
\newcommand\gistlink[2]{\href{https://gist.github.com/oliverdixon/#1}{#2}}
\newcommand\weblink[2]{\href{https://www-users.york.ac.uk/~od641/#1}{#2}}

% \midtilde: draw a vertically centered text-mode tilde
\newcommand\midtilde{\raisebox{.5ex}{\texttildelow}}

% \textline: draw a black horizontal rule across the writeable page width
\newcommand\textline{\par\rule{\textwidth}{.4pt}}

% \socialicon: prints a small social media icon
%
%   #1: the icon file;
%   #2: the amount by which the icon should be raised.
%
\newcommand\socialicon[2]
  {\raisebox{-#2}{\includegraphics[width=.8em]{icons/#1}}}

\begin{document}
\phantomsection%
\makeatletter
\addcontentsline{toc}{chapter}{\@author: \@title}%
\makeatother
\begin{center}
  \begingroup
  \sffamily\large
  \makeatletter
  \setlength\tabcolsep{0pt}
  \renewcommand\arraystretch{1.3}
  \begin{tabularx}{\textwidth}{Xr@{\hspace{10pt}}c}
    \Large\textbf{\@author} &
      \href{https://github.com/oliverdixon}{github.com/oliverdixon} &
      \socialicon{github}{1pt} \\ &
    \href{https://www-users.york.ac.uk/~od641}
      {www-users.york.ac.uk/\midtilde{}od641} &
      \socialicon{internet}{1pt} \\
    7 Earle Street, York, YO31 7ES &
      \href{tel:+447341416944}{+44 (0) 7341416944} &
      \socialicon{telephone}{1.5pt} \\
    Mathematics \& Computer Science Student &
      \href{mailto:od641@york.ac.uk}{od641@york.ac.uk} &
      \socialicon{email}{-.5pt}
  \end{tabularx}
  \makeatother
  \endgroup
\end{center}
\textline%
\phantomsection%
\addcontentsline{toc}{section}{Summary}%
\vspace{2em}
I am a Mathematics and Computer Science joint honours undergraduate student at
the University of York, currently seeking a Year in Industry while averaging
over 80\% in academic disciplines. In addition to having a keen and
long-established interest in Mathematics and Theoretical Computer Science,
I am well-versed in a diverse range of practical technologies including C, C++,
Java, Haskell, \LaTeX{}, SQL, Python, and a variety of other scripting
languages. These act in complement to my demonstrable abilities in strong
leadership, communication, and organisation. I am enthusiastic to participate
in a high-pressure and client-oriented environment, and I anticipate to exploit
my problem-solving and analytical skills, gained through the study of
Mathematics and a decade of software-development experience, to devise efficient
solutions that satisfy clients' criteria.

\section{Education}
\begin{itemize}
  \item \textbf{2022--Present, University of York (MMath)}
  \begin{itemize}
    \item MMath Mathematics and Computer Science, with a Year in
      Industry.\par\emph{In progress, with an expected graduation date of
      September 2027. Some first-year grades are currently pending due to the
      ongoing Marking and Assessment boycott.}
    \item Computer Science: Software Part 1: 95\%; Theory: 88\%;
      Software Part 2: 67\%.
    \item Mathematics: Mathematical Skills: 81\%; Calculus: TBC;
      Algebra: 67\%.
  \end{itemize}
  \item \textbf{2018--2020, Wakefield College (A-Levels)}
  \begin{itemize}
    \item A-level Mathematics (\textbf{A})
    \item A-level Further Mathematics (\textbf{A})
    \item A-level Computer Science (\textbf{A})
    \item A-level History (\textbf{A})
  \end{itemize}
  \item \textbf{2015--2018, Outwood Grange Academy (GCSEs)}
  \begin{itemize}
    \item 10 GCSEs achieved at grades \textbf{A}--\textbf{C}, including English
    and Mathematics.
  \end{itemize}
\end{itemize}

\section{Experience and Technical Skills}
Whilst I am constantly concerned with the continual development of my commercial
and managerial aptitude, a wide variety of experiences has allowed me to
ascertain a significant set of skills bound to be invaluable to any employer.
\begin{itemize}
  \item Expertise---substantiated by over a decade's experience---in the C, C++,
  and Java programming languages, the latter of which have allowed me to form a
  solid understanding of modern object-oriented design patterns. Some
  expositions of relevant work can be found on GitHub and my University website;
  brief descriptions are provided below.
  \begin{itemize}
    \item \weblink{pss}{Programming Support Sessions}: During my tenure as the
    Technical Director of the University of York Electronics Society
    (\textit{EngSoc}, formerly \textit{ShockSoc}) throughout 2022, I compiled
    and delivered a full course of the C programming language to large cohorts
    of undergraduate students studying various branches of Engineering.

    During the orchestration of this project, in the interests of securing a
    successful execution, I demonstrated impeccable long-term organisational and
    leadership skills: this included administration of the Societal and
    Departmental assets, and management of all interested parties.

    \item \githublink{calculator-demo}{A Work-In-Progress Computer Algebra
    System}: This project unifies efficient and refined implementations of
    algorithms that are integral mathematical computing applications.  This
    codebase is under active development, with the eventual goal of producing a
    comprehensive system capable of general technical computing.

    \item \weblink{fractal}{A Hardware-Accelerated Fractal Renderer}: This
    product, coupled with a large report, was written for an A-level Computer
    Science qualification, however continues to be maintained in a limited
    capacity. Using C, OpenGL, and assembly language on a 64-bit Linux platform,
    the final product was capable of rendering parametrisable fractals using
    distributed computing paradigms.

    \item \gistlink{}{GitHub Gists Library}: A collection of modular and
    extensible implementations of canonical algorithms and abstract data
    structures which pervade classical Computer Science and Software
    Engineering. Selected contributions include a Hamming Weight benchmarking
    toolkit, an optimised implementation of a string-formatter, and a visual
    demonstration of dynamic allocation, intended for the education of
    undergraduate Computer Science and Electronic Engineering students.
  \end{itemize}
  \item Through my long involvement in open-source communities, and working in
  University special-interest societies, I have established an understanding of
  foundational team-based development processes including agile and continuous
  techniques. I strongly value, and continually endeavour to improve, the
  qualities of efficient communication amongst small teams when working
  collaboratively: this includes experience with unit testing in a large
  selection of languages, including for projects using multiple stacks requiring
  uniform testing.
  \item A matured acquaintance with the \TeX\ and \LaTeX\ typesetting systems.
  In addition to the aforementioned Programming Support Session scripts, which
  were wholly set with \LaTeX, I have an array of similarly technical
  document-preparation projects:
  \begin{itemize}
    \item \githublink{MS1GP}{\textit{Mathematical Skills} Group Project}: A
    small set of documents covering Category Theory, with a particular focus on
    its parallels to functional programming. The repository includes a report
    and large presentation, throughout which abstract concepts are substantiated
    with concrete examples in the Haskell programming language. A collection of
    Bash shell scripts was also developed to facilitate effective management and
    deployment of this project.

    \item \githublink{cv}{Curriculum Vitae}: The sources of my CV demonstrate
    understanding of imperative and declarative design principles for
    non-trivial typesetting tasks.
  \end{itemize}
  \item A familiarity with the Wolfram \textit{Mathematica} technical
  computing system, particularly in the research areas of Automated
  Theorem-Proving and Distributed Workflows. A small cache of
  \textit{Mathematica} notebooks can be found on
  \weblink{misc\_mma}{my University web instance}.
\end{itemize}

\section{References}
References are available upon request.
\end{document}

