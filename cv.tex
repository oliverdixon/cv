% Oliver Dixon's Curriculum Vitae (Public Copy)
% OWD, 2023

% TODO: Add module grades before submitting to FastTrack.
% TODO: My address will probably have changed by the time I start applying.
% TODO: Write the summary

% \loop\iftrue
%         \errmessage{Are you sure that you want to TeX my CV? :)}%
%         \repeat
% \pausing1

\documentclass{article}

\usepackage[a4paper]{geometry} % Use correct margin sizes for A4 paper
\usepackage{setspace} % Custom line spacing
\usepackage[en-GB]{datetime2} % Regional formatting of \today
\usepackage{graphicx} % Graphics support for the social icons
\usepackage{fancyhdr, lastpage} % Header and footer formatting
\usepackage[
        colorlinks = true,
        allcolors  = blue
]{hyperref}

\setstretch{1.1} % Marginally greater line spacing improves readability

% Do not indent new paragraphs. This is usually poor style, but since we are
% preferring lists (which have their own vertical spacing) over paragraphs for
% separation of prose, a non-zero \parindent would require excessive \noindent
% invocations.
\setlength{\parindent}{0pt}

\newlength{\socialsep}
\setlength{\socialsep}{.3\baselineskip}

% Included social icons may have a page group defined; these will not clash with
% the master document page groups, so it is safe to suppress this warning.
\pdfsuppresswarningpagegroup=1

% Remove number printing; preserve the hyperref index and TOC
\renewcommand{\thesection}{\hspace{-1em}}

% Make bullet points slightly smaller
\renewcommand{\labelitemi}{$\vcenter{\hbox{\tiny$\bullet$}}$}

% The 'body' style decorates all pages past the first page.
\renewcommand{\footrulewidth}{.4pt}
\makeatletter
\fancypagestyle{body}{%
	\fancyhf{}
        \renewcommand{\headrulewidth}{\footrulewidth}
        \fancyhead[L]{\@title}
        \fancyhead[R]{\yorkemail{\@author}{od641}}
        \fancyfoot[L]{\today}
        \fancyfoot[R]{Page \thepage{} of~\pageref{LastPage}}
}
\makeatother

% The 'title' style decorates only the title page; only the header is omitted.
\fancypagestyle{title}{%
        \fancyhead{}
        \renewcommand{\headrulewidth}{0pt}
}

\title{Curriculum Vitae}
\author{Oliver Dixon}

% \sociallink: include a social link with a small icon
%
%       #1: The platform identifier, used for resolving the icon file path;
%       #2: The display text for the hyperlink;
%       #3: The hyperlink destination.
%
\newcommand{\sociallink}[3]{%
        \raggedleft%
        {%
                \large
                \href{#3}{#2}%
                \hspace{.8em}%
                \begingroup
                \setbox0=\hbox{\includegraphics[width=.8em]{icons/#1}}%
                \parbox{\wd0}{\box0}
                \endgroup
        }%
        \par
}

% \yorkemail: expand to a clickable mailto URL of a York e-mail address
%
%       #1: The display name;
%       #2: The York username.
%
\newcommand{\yorkemail}[2]{%
        #1
        \textless%
        \href{mailto:#2@york.ac.uk}{\texttt{#2@york.ac.uk}}%
        \textgreater%
}

% \githublink: expand to a clickable link to the ... GitHub ...\ site
% \gistlink  : ... GitHub Gist ...
% \weblink   : ... York HTTP instance ....
%
%       #1: The URL suffix;
%       #2: The display name.
%
\newcommand{\githublink}[2]{\href{https://github.com/oliverdixon/#1}{#2}}
\newcommand{\gistlink}[2]{\href{https://gist.github.com/oliverdixon/#1}{#2}}
\newcommand{\weblink}[2]{\href{https://www-users.york.ac.uk/~od641/#1}{#2}}

% \midtilde: display a vertically centered text-mode tilde
\newcommand{\midtilde}{\raisebox{.5ex}{\texttildelow}}

% \textline: draw a black horizontal rule across the writeable page width
\newcommand{\textline}{\par\rule{\textwidth}{\footrulewidth}}

\begin{document}
%
\thispagestyle{title}
\pagestyle{body}
%
% MAJOR SECTION: TITLE AND METADATA HEADER
%
\begin{center}
        \Large
        \makeatletter
        \textbf{\@author} \hfill
        \sociallink{github}{github.com/oliverdixon}%
                {https://github.com/oliverdixon}
        \makeatother
        \large

        \vspace{\socialsep}
        \hfill \sociallink{internet}{www-users.york.ac.uk/\midtilde{}od641}%
                {https://www-users.york.ac.uk/~od641}

        \vspace{\socialsep}
        7 Monkton Road, York, YO31 9AJ \hfill
        \sociallink{telephone}{+44 (0) 7341416944}{tel:+447341416944}

        \vspace{\socialsep}
        Mathematics \& Computer Science Student \hfill \hspace{5pt}
        \sociallink{email}{od641@york.ac.uk}{mailto:od641@york.ac.uk}
\end{center}
\textline%
%
% MAJOR SECTION: SUMMARY
%
\section{Summary}
TODO

%
% MAJOR SECTION: EDUCATION
%
\section{Education}
\begin{itemize}
        \item \textbf{2022--Present, University of York (MMath)}
        \begin{itemize}
                \item MMath Mathematics and Computer Science, with a Year in
                Industry.

                \textit{In progress, with an expected graduation date of
                September 2027.}

                \item Individual module marks are to be confirmed.
        \end{itemize}
        \item \textbf{2018--2020, Wakefield College (A-Levels)}
        \begin{itemize}
                \item A-level Mathematics (\textbf{A})
                \item A-level Further Mathematics (\textbf{A})
                \item A-level Computer Science (\textbf{A})
                \item A-level History (\textbf{A})
        \end{itemize}
        \item \textbf{2015--2018, Outwood Grange Academy (GCSEs)}
        \begin{itemize}
                \item 10 GCSEs achieved at grades \textbf{A}--\textbf{C},
                including English and Mathematics.
        \end{itemize}
\end{itemize}

%
% MAJOR SECTION: EXPERIENCE, TECHNICAL SKILLS, AND RELATED ACTIVITIES
%
\section{Experience, Technical Skills, and Related Activities}
Whilst I am constantly concerned with the continual development of my commercial
and managerial aptitude, a wide variety of experiences have allowed me to
ascertain a significant set of skills bound to be invaluable to any employer:
\begin{itemize}
        \item Expertise---substantiated by over a decade's experience---in the
        \textit{C} programming language. Some expositions of relevant work
        can be found on GitHub and my University website:
        \begin{itemize}
                \item \weblink{pss}{Programming Support Sessions}: During my
                tenure as the Technical Officer of The University of York
                Electronics Society (\textit{ShockSoc}) in the Autumn and Spring
                terms of 2022, I compiled and delivered a full course of the
                \textit{C} programming language to cohorts of undergraduate
                students studying Electronic Engineering.

                \item \githublink{calculator-demo}{An Infix Arithmetic
                Expression Evaluator}: This project includes a reference
                implementation of the Shunting Yard algorithm, stack-based
                evaluation, and various other commonly techniques including
                memory management and the implementation of abstract data types.
                Implementations in \textit{C} and \textit{Java} are included.

                \item \weblink{fractal}{A Hardware-Accelerated Generalised
                Fractal Renderer}: This product, coupled with a large report,
                was written for an A-Level Computer Science qualification,
                however continues to be maintained in a limited capacity. Using
                C, OpenGL, and assembly language on a 64-bit Linux platform, the
                final product was capable of rendering highly customisable
                self-similar two-dimensional fractals to an \textit{X11} window
                using distributed computing paradigms.

                \item \gistlink{}{GitHub Gists Library}: A collection of modular
                and extensible implementations of canonical algorithms and
                abstract data structures which pervade classical Computer
                Science and Software Engineering.  Selected contributions
                include a
                \gistlink{4cb5cf0f918957ca0cd35306635772cc}{high-performance
                \texttt{popcount} (zero-counting) benchmarking toolkit}, and
                \gistlink{b0c4984b7004518309bb67d6c5f6ae7d\#file-sztoa-c}{an
                optimised implementation of a string-formatter}, emulating the
                behaviour of a complex and heavyweight \texttt{printf} standard
                library feature.
        \end{itemize}
        \item A matured acquaintance with the \TeX\ and \LaTeX\ typesetting
        systems. In addition to the \weblink{pss}{aforementioned Programming
        Support Session scripts}, which were wholly set with \LaTeX, I have an
        array of similarly technical document-preparation projects:
        \begin{itemize}
                \item \githublink{MS1GP}{\textit{Mathematical Skills} Group
                Project}: A small set of documents covering Category Theory,
                with a particular focus on its parallels to functional
                programming and The $\lambda$-Calculus. The repository
                principally includes a
                \githublink{MS1GP/blob/master/report/report.tex}{report} and
                large
                \githublink{MS1GP/blob/master/presentation/presentation.tex}%
                        {\textit{Beamer} presentation}. Throughout both mediums,
                abstract concepts are substantiated with concrete examples in
                the purely functional \textit{Haskell} programming language, in
                which I am also fluent. A collection of \textit{Bash} shell
                scripts was also developed to facilitate effective management
                and deployment of this project.

                \item \githublink{cv}{Curriculum Vitae}: The sources of my CV
                demonstrate understanding of imperative and declarative design
                principles for non-trivial typesetting tasks.

                \item \githublink{opium}{A Humanities and Social Sciences
                Essay}: Having conceptualised and developed a large essay for an
                A-Level \textit{AQA History} qualification, I also possess
                reasonable experience with typesetting large amounts of
                non-technical prose for professional delivery; this includes the
                setting of non-Latin scripts including Standard Chinese and
                Cyrillic.
        \end{itemize}
        \item A familiarity with the \textit{Wolfram Mathematica} technical
        computing system, particularly in the research areas of Automated
        Theorem-Proving and Distributed Workflows. A small cache of
        \textit{Mathematica} notebooks, published to demonstrate moderate
        competence in the~\textit{Wolfram Language}, can be found on
        \weblink{misc\_mma}{my University web instance}.
\end{itemize}
%
% MAJOR SECTION: REFERENCES
%
\section{References}
References are available on request.
%
\end{document}

